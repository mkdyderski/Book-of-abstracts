\providecommand{\main}{..} 
\documentclass[\main/boa.tex]{subfiles}

\begin{document}

\section{Turning Text Mining into Language Mining: Corpus Linguistics in R}

\begin{center}
  {\bf \index[a]{George Moroz}$^{1^\star}$}
\end{center}

\vskip 0.3cm

\begin{affiliations}
\begin{enumerate}
\begin{minipage}{0.915\textwidth}
\centering
\item School of Linguistics, National Research University Higher School of
Economics \\[-2pt]
\end{minipage}
\end{enumerate}
$^\star$Contact author: \href{mailto:agricolamz@gmail.com}{\nolinkurl{agricolamz@gmail.com}}\\
\end{affiliations}

\vskip 0.5cm

\begin{minipage}{0.915\textwidth}
\keywords corpus linguistics; cross-linguistic linked databases; minority language
analysis; text mining
\packages shiny; rmarkdown; leaflet; stringr
\end{minipage}

\vskip 0.8cm

Few books deal with both linguistics and R. Most of them are actually
textbooks on statistics with examples of real-life linguistic problems
solved. So it is about analyzing linguistic data in \textbf{R} (there is
nothing special about it, statistically speaking), not about doing
linguistics in \textbf{R}. Fortunately, there are some linguistic
packages in \textbf{R}. Santiago Barreda created package
\emph{phonTools} for phonetics research and experiments. Aaron Albin
made a package \emph{PraatR}, which provides \textbf{R} with the
functionality of the well known phonetic program Praat by Paul Boersma
and David Weenink. There are some text mining technics implemented in
\textbf{R} (e.g. \emph{tm}) and a few packages which allows extracting
data from social networks.

However, linguistic \textbf{R} packages are mostly focused on phonetics.
Pure text mining is insufficient for the linguistic research. So we
decided to create an extended package, which helps to solve problems in
linguistic typology and minority language description. In this talk, I
will show preliminary results dealing with creation of a language
corpora using \emph{shiny}, \emph{rmarkdown}, \emph{leaflet}, and
\emph{stringr}. Problems, that I will cover during the talk:

\begin{itemize}
\tightlist
\item
  storing and analyzing texts in minority language: transcription,
  translation and morphological analysis; all examples are based on
  Adyghe (Circassian language family) and Mehweb (Northeast Caucasian
  language family)
\item
  standardization of language documentation in \textbf{R}: integration
  with Cross-Linguistic Linked Databases
\end{itemize}

Support from the Basic Research Program of the National Research
University Higher School of Economics is gratefully acknowledged

\end{document}
