\providecommand{\main}{..} 
\documentclass[\main/boa.tex]{subfiles}

\begin{document}

\section{Latent Class Analysis in Psychology}

\begin{center}
  {\bf \index[a]{Paweł Kleka}$^{1^\star}$}
\end{center}

\vskip 0.3cm

\begin{affiliations}
\begin{enumerate}
\begin{minipage}{0.915\textwidth}
\centering
\item Institute of Psychology, Adam Mickiewicz University \\[-2pt]
\end{minipage}
\end{enumerate}
$^\star$Contact author: \href{mailto:pawel.kleka@amu.edu.pl}{\nolinkurl{pawel.kleka@amu.edu.pl}}\\
\end{affiliations}

\vskip 0.5cm

\begin{minipage}{0.915\textwidth}
\keywords LCA; classification;
\packages e1071
\end{minipage}

\vskip 0.8cm

In my opinion, the methods of statistical analysis are not developing in
psychology, it is happend just beyond. Then, methods are only used by
psychologists on their home ground. Such a method, useful for building
typology, is latent classes analysis (LCA), which I had `discovered' by
accident. In my presentation, I would like to show how use LCA to
typical psychological data, which rarely meet the requirements of
parametric methods (quantitative measuring scale, distribution
normality, no outliers).

\end{document}
