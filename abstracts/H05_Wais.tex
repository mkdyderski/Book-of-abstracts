\providecommand{\main}{..} 
\documentclass[\main/boa.tex]{subfiles}

\begin{document}

\section{LimeRick: Bridge between LimeSurvey and R}

\begin{center}
  {\bf \index[a]{Kamil Wais}$^{1^\star}$}
\end{center}

\vskip 0.3cm

\begin{affiliations}
\begin{enumerate}
\begin{minipage}{0.915\textwidth}
\centering
\item University of Information Technology and Management in Rzeszów \\[-2pt]
\end{minipage}
\end{enumerate}
$^\star$Contact author: \href{mailto:kamil.wais@gmail.com}{\nolinkurl{kamil.wais@gmail.com}}\\
\end{affiliations}

\vskip 0.5cm

\begin{minipage}{0.915\textwidth}
\keywords LimeSurvey, on-line survey, CAWI, survey meta data
\packages LimeRick (not publicly available yet)
\end{minipage}

\vskip 0.8cm

The first public presentation of the prototype of a new \textbf{R}
package that provides useful connection between \textbf{R} and the most
popular open-source web scripts for on-line surveys
(\url{http://www.LimeSurvey.org}).

The package aims to enable and simplify the work flow of reproducible
CAWI research in Social Science; preparing templates for ad hoc and
real-time analysis, performing detailed meta-analysis, archiving and
monitoring responses directly from \textbf{R}.

With the \emph{LimeRick} package one can import pre-processed data from
an on-line survey with an \textbf{R} script (with the use of
RemoteConrol2 API or with non-API solution). Then, the data can be
processed analytically by an \textbf{R} script and automatically
reported with the use of \emph{shiny} package. The whole process can be
pre-programmed and performed without the researcher interference. This
enables to build data products based on declarative data from on-line
surveys, which are processed, analyzed, and visualized on-line and in
real-time. It can be of use for performing automatic tracking studies
with real-time KPIs monitoring (e.g.~for a customer satisfaction
survey). The package will be also linked to a tool for unique, advanced
analysis of meta-data from an on-line survey, which has already been
alpha tested during large CAWI research project. As the early prototype
of the package is presented, the presentation will end with a call for
collaboration for developing and testing the package in different usage
cases.

\end{document}
