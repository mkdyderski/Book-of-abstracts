\providecommand{\main}{..} 
\documentclass[\main/boa.tex]{subfiles}

\begin{document}

\section{Exploratory data analysis of a clinical study group - revealing patient
subgroups.}

\begin{center}
  {\bf \index[a]{Bogumil M. Konopka}$^{1^\star}$, \index[a]{Felicja Lwow}$^{2}$, \index[a]{Łukasz Łaczmański}$^{3}$}
\end{center}

\vskip 0.3cm

\begin{affiliations}
\begin{enumerate}
\begin{minipage}{0.915\textwidth}
\centering
\item Wroclaw University of Science and Technology \\[-2pt]
\item The University School of Physical Education in Wroclaw \\[-2pt]
\item Wroclaw Medical University \\[-2pt]
\end{minipage}
\end{enumerate}
$^\star$Contact author: \href{mailto:bogumil.konopka@pwr.edu.pl}{\nolinkurl{bogumil.konopka@pwr.edu.pl}}\\
\end{affiliations}

\vskip 0.5cm

\begin{minipage}{0.915\textwidth}
\keywords clustering; outlier detection; PCA
\packages clv; ggplot2; chemometrics
\end{minipage}

\vskip 0.8cm

Thorough knowledge of the structure of analyzed data allows to form
precise research questions. The data structure and basic associations
between parameters in the data can be revealed with methods for
exploratory data analysis. Currently a researcher has a whole plethora
of exploratory tools to choose from. Selecting methods that will work
together well and facilitate data interpretation is not an easy task. In
this work we present a well fitted set of tools for a complete
exploratory analysis of a clinical study dataset and perform a case
study analysis on a set of 515 patients. The proposed procedure
comprises several steps:

\begin{enumerate}
\def\labelenumi{\arabic{enumi}.}
\tightlist
\item
  robust data normalization,
\item
  outlier detection with Mahalanobis (MD) and robust Mahalanobis
  distances (rMD)
\item
  hierarchical clustering with Ward's algorithm,
\item
  Principal Component Analysis with biplot vectors.
\end{enumerate}

Introductory analysis showed that the case-study dataset comprises two
clusters separated along the axis of sex hormone attributes. Further
analysis was carried out separately for male and female patients. The
most optimal partitioning in the male set resulted in five subgroups.
Two of them were related to diseased patients: diabetes and gonadotroph
adenoma patients. Analysis of the female set suggested that it was more
homogeneous than the male dataset. No evidence of pathological patient
subgroups were found. In the study we showed that outlier detection with
MD and rMD allows not only to identify outliers but can also assess the
heterogeneity of a dataset. The case study proved that our procedure is
well suited for identification and visualization of biologically
meaningful patient subgroups.

\end{document}
